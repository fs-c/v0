\documentclass{article}
\usepackage[utf8]{inputenc}

\title{Übung 3}
\author{Laurenz Weixlbaumer, 11804751}
\date{November 2018}

\renewcommand{\arraystretch}{1.3}
\renewcommand\thesubsection{(\alph{subsection})}

\usepackage{enumitem}
\usepackage{mathtools}

\begin{document}

\maketitle

\section{Binärzahlen}

\begin{enumerate}[label=(\alph*)]
\item Tabelle mit verschiedenen binären Darstellungsarten.
\begin{center}
\begin{tabular}{ | r | r | r | r | r | }
    \hline
    Dezimal & Betrag/Vorzeichen & 1er-Komplement & 2er-Komplement & Offset_{16}\\
    \hline \hline
    $7_{10}$ & $00111_2$ & $00111_2$ & $00111_2$ & $10111_2$ \\
    \hline
    $-9_{10}$ & $11001_2$ & $10110_2$ & $10111_2$ & $00111_2$ \\
    \hline
    $11_{10}$ & $01011_2$ & $01011_2$ & $01011_2$ & $11011_2$ \\
    \hline
    $5_{10}$ & $00101_2$ & $00101_2$ & $00101_2$ & $10101_2$ \\
    \hline
    $-11_{10}$ & $11011_2$ & $10100_2$ & $10101_2$ & $00101_2$ \\
    \hline
\end{tabular}
\end{center}

\item \dots

$-77.625_{10}$ als nicht vorzeichenbehaftete Binärzahl ist $01001101.101_2$.

\begin{center}
\begin{tabular}{ c | c }
    $01001101.101_2$ & negieren \\
    $10110010.010_2$ & +1 \\
    \hline \hline
    $10110010.011_2$ & 2er-Komplement
\end{tabular}
\end{center}

\item \dots

\begin{center}
\begin{tabular}{ c | c }
    $11781 + (- 16223)$ & Umwandlung zu 10er-Komplement \\
    $11781 + 83777$ & Addition \\
    $95558$ & Rückwandlung \\
    \hline \hline
    $-04442$ & Ergebnis in dezimal
\end{tabular}
\end{center}

\item $1111_2 + 1111_2$ führt zu $10000_2$. Nachdem aber nur 4 Bits pro Zahl zur Verfügung stehen, wird das tatsächliche Ergebnis wahrscheinlich $0000_2$ betragen -- es kommt zu einer Bereichsüberschreitung.

Ist der Überlauf nicht $0_2$, ist es durch eine Addition zu einer Bereichs\-überschreitung gekommen.

\end{enumerate}

\end{document}
